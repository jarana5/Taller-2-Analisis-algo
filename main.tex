%% LyX 2.3.2 created this file.  For more info, see http://www.lyx.org/.
%% Do not edit unless you really know what you are doing.
\documentclass[twoside,spanish]{elsarticle}
\usepackage[T1]{fontenc}
\usepackage[latin9]{inputenc}
\pagestyle{headings}
\usepackage{float}
\usepackage{amssymb}
\PassOptionsToPackage{normalem}{ulem}
\usepackage{ulem}

\makeatletter

%%%%%%%%%%%%%%%%%%%%%%%%%%%%%% LyX specific LaTeX commands.
\floatstyle{ruled}
\newfloat{algorithm}{tbp}{loa}
\providecommand{\algorithmname}{Algoritmo}
\floatname{algorithm}{\protect\algorithmname}

%%%%%%%%%%%%%%%%%%%%%%%%%%%%%% User specified LaTeX commands.
\usepackage{algorithm}
\usepackage{algpseudocode}

% specify here the journal
\journal{Curso de <<Analisis de algoritmos>>, PUJ, Bogota, Colombia - }

% use this if you need line numbers

\makeatother

\usepackage{babel}
\addto\shorthandsspanish{\spanishdeactivate{~<>}}

\begin{document}

\begin{frontmatter}{}

\title{Taller 2}

\tnotetext[t1]{En este documento se realiza el taller 2 de analisis de algoritmos, en el cual se nos pidio elaborar un algoritmo de dividir y vencer para hacer un juego de adivina el numero donde el usuario lo asigna y el algoritmo lo adivina.}

\author[lfv]{Julian Arana}

\ead{juliana-aranag@javeriana.edu.co}


\address[lfv]{Pontificia Universidad Javeriana, Bogota, Colombia}
\begin{abstract}
En este algoritmo veremos el desarrollo del algoritmo que permite hacer el juego de adivina el numero donde el usuario lo asigna y el algoritmo lo adivina.
\end{abstract}
\begin{keyword}
algoritmo, escritura formal, adivinar, dividir y vencer.
\end{keyword}

\end{frontmatter}{}

\section{Analisis del problema}

En este problema tenemos 2 actores, el software dara un numero aleatorio entre 1 y 1000. El numero que de sera el limite superior que se tendra, por ejemplo si el numero es 500 el rango del numero que el usuario puede poner es de 1 a 500.

Un ejemplo de como deberia funcionar:

\[
|1|2|3|4|5|6|7|8|9|10|
\]

Numero a adivinar: 8.

Dividir el algoritmo a la mitad, si el numero a adivinar es menor al de adivinar, se usara como limite inferior el numero en la mitad, si es al reves, se usa este numero como el limite superior.

\[
|5|6|7|8|9|10| \rightarrow |8|9|10| \rightarrow |8|9| \rightarrow |8|
\]

Numero encontrado.

\section{Diseno del problema}

El annlisis anterior nos permite disenar el problema: definir las
entradas y salidas de un posible algoritmo de solucion, que aun no
esta definido.
\begin{enumerate}
\item \emph{\uline{Entradas}}: Los limites superior [e] e inferior [b] \rightarrow [b,e], El numero a adivinar a.

\item \emph{\uline{Salidas}}: Un mensaje diciendo si el numero adivinado es mayor, menor o igual al ingresado por el usuario.
\end{enumerate}

\section{Algoritmos de solucion}

\subsection{Algoritmo evidente}

Este algoritmo de solucion es una traduccion literal de las deficiones
de lo que se quiere resolver.

\begin{algorithm}[H]
    \caption{Algoritmo de búsqueda binaria}
    \begin{algorithmic}[1]
        \Procedure{enc\_aux}{$b, e, a$}
            \If{$b > e$}
                \State \Return $0$
            \Else
                \State $q \leftarrow \text{round}((b + e) / 2)$
                \State \textbf{print} "El número adivinado por el algoritmo es: $q$"
                \If{$q > a$}
                    \State \textbf{print} "El número es menor"
                    \State \Return \textsc{enc\_aux}($b, q, a$)
                \ElsIf{$q < a$}
                    \State \textbf{print} "El número es mayor"
                    \State \Return \textsc{enc\_aux}($q, e, a$)
                \Else
                    \State \textbf{print} "El número es correcto"
                    \State \Return $q$
                \EndIf
            \EndIf
        \EndProcedure
    \end{algorithmic}
\end{algorithm}





\begin{algorithm}[H]
    \caption{Función para encontrar el número}
    \begin{algorithmic}[1]
        \Procedure{encontrar}{}
            \State $rango \gets \text{random.randint}(1, 1000)$
            \State \textbf{print} "El rango del numero va desde 0 a $rango$"
            \State $numero \gets \text{int(input}("Ingrese el numero a adivinar"))$
            \While{$numero > rango$}
                \State $numero \gets \text{int(input}("El numero ingresado es mayor al rango dado. Seleccione otro número"))$
            \EndWhile
            \State \text{adivinado} $\gets \text{enc\_aux}(0, rango, numero)$
        \EndProcedure
    \end{algorithmic}
\end{algorithm}




\end{document}
